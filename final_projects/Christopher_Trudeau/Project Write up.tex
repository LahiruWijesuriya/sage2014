\documentclass{article}
\title{Final Project Write Up}
\author{Christopher Trudeau}

\begin{document}
\maketitle
\newpage
\begin{flushleft}
Please visit https://778f28d3.ngrok.com/ to find my project. If for any reason the link does not work, please email me at trudeau.cj@gmail.com and I will fix it asap! Some small notes, please read the Grader README.md. To get access to the GitHub Repo if you would like please email me at trudeau.cj@gmail.com. If you are running MacOSX or Linux there are setup instructions on how to get my project running locally and how to contribute to my project it will explain to you how everything works. The goal of my project was to give the ability to any developer access to the power that is inside of SageMathCloud and harness the power of doing Math with Sage in the Cloud programmatically by adding documentation for the API in SageMathCloud. Using this documentation anyone could make an application and leverage Sage to do any math programming without the need to implement the math programming themselves. 
\end{flushleft}
\begin{flushleft}
My project is a static website that can be easily expanded using the MarkDown format as seen in the Grader README.md. The goal was to make it as easy as possible for anyone with or without programming knowledge they could add to the documentation of SageMathCloud if they desired. An API is an Application Programming Interface that allows a product or service to interact with another product or service programmatically. The point of the API is to help leverage the same logic for all of your applications, partnerships with others or business, data exchange and monetization. The documentation for the API is to use the JSON format. In the Grader README.md I have added a link to the GitHub Repository of all of the code, but it is also in the directory locally because I can not expose the GitHub Project yet because Professor Stein wants to keep it private. Therefore I have put a copy locally here if you would like to download and use it. 
\end{flushleft}
\begin{flushleft}
The project directory also has my presentation Project Presentation rendered.sagews that contains my presentation from Friday. The reason it is in markdown is because I included it inside my project. 
\end{flushleft}
\end{document}
