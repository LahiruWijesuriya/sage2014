\documentclass{beamer}
\usepackage[latin1]{inputenc}
\usepackage{graphicx}
\usepackage{fixltx2e}
\usetheme{Warsaw}

\title{Sudokus!}
\author{Gage Irving}\institute{University of Washington}

\begin{document}

\begin{frame}
\titlepage
\end{frame}

\begin{frame}
\frametitle{What is a sudoku?}
\begin{array}{|*{3}{*{3}{r}|}}\hline\n5&1&3&6&8&7&2&4&9\\\n8&4&9&5&2&1&6&3&7\\\n2&6&7&3&4&9&5&8&1\\\hline\n1&5&8&4&6&3&9&7&2\\\n9&7&4&2&1&8&3&6&5\\\n3&2&6&7&9&5&4&1&8\\\hline\n7&8&2&9&3&4&1&5&6\\\n6&3&5&1&7&2&8&9&4\\\n4&9&1&8&5&6&7&2&3\\\hline\n
\end{array}
\\
The objective is to fill a D X D, where D is a perfect square (ie D = d\textsuperscript{2} where d is an integer, most often D = 9 and d = 3), grid with digits so that each column, each row, and each of the D \sqrt{D} X \sqrt{D} sub\-squares that compose the Sudoku grid contains all of the digits from 1 to D.
\end{frame}





\end{document}
%sagemathcloud={"zoom_width":155}
